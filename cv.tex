%%%%%%%%%%%%%%%%%%%%%%%%%%%%%%%%%%%%%%%%%
% Wilson Resume/CV
% XeLaTeX Template
% Version 1.0 (22/1/2015)
%
% This template has been downloaded from:
% http://www.LaTeXTemplates.com
%
% Original author:
% Howard Wilson (https://github.com/watsonbox/cv_template_2004) with
% extensive modifications by Vel (vel@latextemplates.com)
%
% License:
% CC BY-NC-SA 3.0 (http://creativecommons.org/licenses/by-nc-sa/3.0/)
%
%%%%%%%%%%%%%%%%%%%%%%%%%%%%%%%%%%%%%%%%%

% Carefully read my old CV, typeset in LaTeX.

%----------------------------------------------------------------------------------------
%	PACKAGES AND OTHER DOCUMENT CONFIGURATIONS
%----------------------------------------------------------------------------------------

\documentclass[10pt]{article} % Default font size

%%%%%%%%%%%%%%%%%%%%%%%%%%%%%%%%%%%%%%%%%
% Wilson Resume/CV
% Structure Specification File
% Version 1.0 (22/1/2015)
%
% This file has been downloaded from:
% http://www.LaTeXTemplates.com
%
% License:
% CC BY-NC-SA 3.0 (http://creativecommons.org/licenses/by-nc-sa/3.0/)
%
%%%%%%%%%%%%%%%%%%%%%%%%%%%%%%%%%%%%%%%%%

%----------------------------------------------------------------------------------------
%	PACKAGES AND OTHER DOCUMENT CONFIGURATIONS
%----------------------------------------------------------------------------------------

\usepackage[a4paper, hmargin=25mm, vmargin=30mm, top=20mm]{geometry} % Use A4 paper and set margins

\usepackage{fancyhdr} % Customize the header and footer

\usepackage{lastpage} % Required for calculating the number of pages in the document

\usepackage[hidelinks]{hyperref} % Colors for links, text and headings

\setcounter{secnumdepth}{0} % Suppress section numbering

%\usepackage[proportional,scaled=1.064]{erewhon} % Use the Erewhon font
%\usepackage[erewhon,vvarbb,bigdelims]{newtxmath} % Use the Erewhon font
\usepackage[utf8]{inputenc} % Required for inputting international characters
\usepackage[T1]{fontenc} % Output font encoding for international characters

\usepackage{fontspec} % Required for specification of custom fonts
\setmainfont[Path = ./fonts/,
Extension = .otf,
BoldFont = Erewhon-Bold,
ItalicFont = Erewhon-Italic,
BoldItalicFont = Erewhon-BoldItalic,
SmallCapsFeatures = {Letters = SmallCaps}
]{Erewhon-Regular}

\usepackage{color} % Required for custom colors
\definecolor{slateblue}{rgb}{0.17,0.22,0.34}

\usepackage{sectsty} % Allows customization of titles
\sectionfont{\color{slateblue}} % Color section titles

%\fancypagestyle{plain}{\fancyhf{}\rfoot{Aug 2020}} % Define a custom page style
\fancypagestyle{plain}{\fancyhf{}\rfoot{\today}} % Define a custom page style
\pagestyle{plain} % Use the custom page style through the document
\renewcommand{\headrulewidth}{0pt} % Disable the default header rule
\renewcommand{\footrulewidth}{0pt} % Disable the default footer rule

\setlength\parindent{0pt} % Stop paragraph indentation

% Non-indenting itemize
\newenvironment{itemize-noindent}
{\setlength{\leftmargini}{0em}\begin{itemize}}
{\end{itemize}}

% Bullets
\usepackage{amssymb}
\renewcommand{\bullet}{{\color{slateblue}\sqbullet}\hspace{0.6em}}
% \let\sqbullet\bullet

% Text width for tabbing environments
\newlength{\smallertextwidth}
\setlength{\smallertextwidth}{\textwidth}
\addtolength{\smallertextwidth}{-2cm}

\newcommand{\sqbullet}{~\vrule height 1ex width .8ex depth -.2ex} % Custom square bullet point definition

%----------------------------------------------------------------------------------------
%	MAIN HEADER COMMAND
%----------------------------------------------------------------------------------------

\renewcommand{\title}[1]{
{\huge{\color{slateblue}\textbf{#1}}}\\ % Header section name and color
\rule{\textwidth}{0.5mm}\\ % Rule under the header
}

%----------------------------------------------------------------------------------------
%	JOB COMMAND
%----------------------------------------------------------------------------------------

\newcommand{\job}[6]{
\begin{tabbing}
\hspace{2cm} \= \kill
\textbf{#1} \> \href{#4}{#3} \\
\textbf{#2} \>\+ \textit{#5} \\
\begin{minipage}{\smallertextwidth}
\vspace{2mm}
#6
\end{minipage}
\end{tabbing}
\vspace{2mm}
}

%----------------------------------------------------------------------------------------
%	SKILL GROUP COMMAND
%----------------------------------------------------------------------------------------

\newcommand{\skillgroup}[2]{
\begin{tabbing}
\hspace{5mm} \= \kill
\sqbullet \>\+ \textbf{#1} \\
\begin{minipage}{\smallertextwidth}
\vspace{2mm}
#2
\end{minipage}
\end{tabbing}
}

%----------------------------------------------------------------------------------------
%	INTERESTS GROUP COMMAND
%-----------------------------------------------------------------------------------------

\newcommand{\interestsgroup}[1]{
\begin{tabbing}
\hspace{5mm} \= \kill
#1
\end{tabbing}
\vspace{-10mm}
}

\newcommand{\interest}[1]{\sqbullet \> \textbf{#1}\\[3pt]} % Define a custom command for individual interests

%----------------------------------------------------------------------------------------
%	TABBED BLOCK COMMAND
%----------------------------------------------------------------------------------------

\newcommand{\tabbedblock}[1]{
\begin{tabbing}
\hspace{2cm} \= \hspace{4cm} \= \kill
#1
\end{tabbing}
} % Include the file specifying document layout
\usepackage[maxnames=4,minnames=4,sorting=ydnt]{biblatex}
\addbibresource{publications.bib}

%----------------------------------------------------------------------------------------

\begin{document}

%----------------------------------------------------------------------------------------
%	NAME AND CONTACT INFORMATION
%----------------------------------------------------------------------------------------

\title{Federico Marotta -- Curriculum Vit\ae} % Print the main header

%------------------------------------------------

\parbox{0.5\textwidth}{ % First block
\begin{tabbing} % Enables tabbing
\hspace{3cm} \= \hspace{4cm} \= \kill % Spacing within the block
{\bf Full Name} \> Federico Marotta \\
{\bf Date of Birth} \> 20\textsuperscript{\textit{th}} January 1996 \\ % Date of birth 
% {\bf Gender} \> Male \\
{\bf Nationality} \> Italian % Nationality
\end{tabbing}}
\hfill % Horizontal space between the two blocks
\parbox{0.5\textwidth}{ % Second block
\begin{tabbing} % Enables tabbing
\hspace{3cm} \= \hspace{4cm} \= \kill % Spacing within the block
{\bf E-mail} \> \href{mailto:federico.marotta@edu.unito.it}{federico.marotta@edu.unito.it} \\
{\bf Personal Website} \> \href{https://www.fmarotta.dynu.net}{https://www.fmarotta.dynu.net} \\
% {\bf Web Page} \> \href{https://www.fmarotta.dynu.net}{https://www.fmarotta.dynu.net} \\
% {\bf Mobile Phone} \> +0 (000) 111 1112 \\ % Mobile phone
\end{tabbing}}


%----------------------------------------------------------------------------------------
%	EDUCATION SECTION
%----------------------------------------------------------------------------------------

\section{Education}

\tabbedblock{
\bf{2021-Present} \> PhD student in Bioinformatics --- \href{https://www.embl.org/groups/bork/}{European Molecular Biology Laboratory} \\[5pt]
\>\+
\bullet Machine Learning \\
\bullet Sequence analysis \\
\bullet Phylogenetics
}

%------------------------------------------------

\tabbedblock{
\bf{2018-2021} \> M.Sc. Stochastics and Data Science (110/110L) --- \href{http://www.unito.it}{Università degli Studi di Torino} \\[5pt]
\>\+
\textbf{Dissertation:} The effect of genetic variants on complex 
diseases (duration: 1 year; 61 pages)\\
\bullet Machine Learning \\
\bullet Frequentist and Bayesian statistics \\
\bullet Probability theory and stochastic processes
}

%------------------------------------------------

\tabbedblock{
\bf{2015-2018} \> B.Sc. Biotechnology (110/110L) --- \href{http://www.unito.it}{Università degli Studi di Torino} \\[5pt]
\>\+
\bullet Molecular and cellular biology \\
\bullet Biochemistry and general chemistry \\
\bullet Bioinformatics skills acquired during internship
}

%----------------------------------------------------------------------------------------
%	EMPLOYMENT HISTORY SECTION
%----------------------------------------------------------------------------------------

\section{Work Experience}

\job
{Mar 2016-}{Apr 2021}
{Computational Biology Unit (Prof. Paolo Provero's laboratory), Turin (Italy)}
{https://proverolab.gitlab.io/}
{Internship student}
{
\bullet Research in Bioinformatics \\
\bullet Data Mining and Programming
}

\job
{Mar 2020-}{Jun 2020}
{Course of ``Elements of physics, computer science and statistics'', University of Turin (Italy)}
{https://www.clproduzionianimali.unito.it/do/storicocorsi.pl/Show?_id=9489_1920}
{Student tutor of computer science and statistics}
{
\bullet Clarification of the students' doubts \\
\bullet Preparation of teaching material
}

\section{Research Activities}

\paragraph{Master thesis research project}

During my master's degree program I started working on a project with the goal of finding genes whose \textit{cis-}regulated expression is associated to a complex disease. This approach, commonly known as a transcriptome-wide association study, exploits a chain of associations: first from genetic variants to gene expression, and then from expression to disease. I wrote code to perform the prediction of gene expression from the DNA sequence, and then to associate the predicted expression to the disease status. The novelty of our approach is that the prediction is based on the affinities of transcription factors for the DNA sequence, rather than on the raw genotypes. The project also involved the reconstruction and analysis of a regulatory network. I implemented an R package (\href{https://cran.r-project.org/web/packages/fplyr/index.html}{fplyr}) to help me performing these analyses.

\paragraph{Previous activities} When I joined Prof. Provero's lab for my bachelor's thesis, I contributed to the existing research projects of the group, focused on the characterisation of SNP-disease association through the effect of the SNP on the affinity of the DNA sequence for transcription factors; I performed some basic computational analyses.

\paragraph{Side projects} I developed an environment to automatically organise the data we download at our lab into appropriate folders, and to manage our research projects, enforcing good practices of workflow development, in a multi-user setting (\href{https://github.com/fmarotta/bioinfoconda}{https://github.com/fmarotta/bioinfoconda}).

%----------------------------------------------------------------------------------------
%	IT/COMPUTING SKILLS SECTION
%----------------------------------------------------------------------------------------

\section{Personal Skills and Competences}

\skillgroup{Human Languages}
{
Italian (Mother tongue) \\
English (C2)
}

\skillgroup{Computer Languages}
{
R, python, C, JavaScript
}

\skillgroup{\LaTeX}
{
I enjoy typesetting in \LaTeX\ and I am the author of a template for books or theses:\\
\href{http://latextemplates.com/template/kaobook/}{http://latextemplates.com/template/kaobook}
}

%\skillgroup{Coding}
%{
%\textit{mySQL}
%}

%\skillgroup{Systems Administration}
%{
%\textit{GNU/Linux}
%}


%----------------------------------------------------------------------------------------
%	INTERESTS SECTION
%----------------------------------------------------------------------------------------

%\section{Interests}

%\interestsgroup{
%\interest{Science}
%\interest{Web Development}
%\interest{Sports}
%\interest{\LaTeX}
%}

%----------------------------------------------------------------------------------------

\nocite{*}
\printbibliography[title=Publications]

\end{document}

% Now read the updated content.

% Federico Marotta, PhD candidate
% Karlsruher Strasse 9, Heidelberg, DE 69126
% Nationality: Italian
% +49 162 583 4769 | fmarotta@proton.me
% GitHub | ORCiD | LinkedIn
% PROFILE
% Bioinformatician with formal education in Applied Mathematics and 7 years of international academic research experience at the intersection of biology, mathematics, and computer science
% Strong foundation in bioinformatics, human genetics, and mathematical modelling.
% Formal experience with DNA/RNA sequencing, design of data analysis workflows, and scientific software development in R, Python, and Julia
% Passionate about developing automations and leveraging technology to streamline complex processes (both in and out of the workplace)
% Personal quote: “What I cannot code, I do not understand”
% TECHNICAL SKILLS
% Programming and software development
% R—data.table, ggplot, shiny, machine learning, 3 own packages on CRAN (8 years)
% Python—data structures and algorithms, multi-threading (6 years)
% Julia—numerical computing, optimization (3 years)
% Pipeline development—Nextflow, Snakemake
% Databases—mongoDB, PostgreSQL
% NGS data analysis—variant calling, variant imputation
% Best practices for robust and reproducible software development—version control, Gitlab CI/CD, documentation and reports
% Linux and high-performance computing
% Proficient in Linux and high-performance computing environments—slurm
% Bioinformatic tools for sequence and structure homology search—hmmer, BLAST, Foldseek
% Mathematics and statistics
% Hypothesis testing and association studies
% Machine learning (supervised and unsupervised)
% Mathematical modelling and stochastic processes
% Domain knowledge
% Biology, statistics, bioinformatics, modelling, data analysis
% WORK EXPERIENCE
% PhD candidate | Bioinformatics and mathematical modelling | May 2021 - Present
% European Molecular Biology Laboratory, Molecular and Systems Biology Unit, Germany
% Performed bioinformatic analyses to identify the function of ~10 previously uncharacterized proteins in a clinically relevant bacterium
% Designed a mathematical model of a biological process, developed an optimization algorithm, and implemented it in high-quality software, able to make testable predictions
% Built workflows with Nextflow and Snakemake for large-scale processing of DNA sequence and protein structure data—metagenomic assembly, similarity search, protein structure prediction
% Collaborated closely with biologists across fields—wet lab, ecology, structural biology
% Graduate researcher | Human genetics | May 2018 - Apr 2021
% University of Turin, Italy
% Processed DNA and RNA sequencing data, studying the effect of genetic variants
% Developed a predictive machine learning model based on DNA sequence
% Devised novel statistical methods to find associations between genes and disease
% Teaching assistant | Introduction to physics and statistics | Mar-Jul 2020
% University of Turin, Italy
% Developed training material, including homework and explanatory visualizations
% Provided support to students by addressing questions and clarifying concepts
% EDUCATION
% PhD | Bioinformatics | May 2021 - Oct 2025 (expected graduation)
% Joint degree European Molecular Biology Laboratory and Heidelberg University, Germany
% Supervisor: Prof. Peer Bork
% Thesis title: Characterization, evolution, and dynamics of cryoET-derived macromolecular assemblies in Mycoplasma pneumoniae
% MSc | Stochastics and Data Science (applied mathematics) | Sep 2018 - Apr 2021
% University of Turin, Italy
% Supervisor: Prof. Paolo Provero, Prof. Silvia Montagna
% Grade 110/110 cum laude
% Thesis title: Bridging the gap between genome, transcriptome, and disease
% BSc | Biotechnology | Sep 2015 - Jul 2018
% University of Turin, Italy
% Grade 110/110 cum laude
% AWARDS
% PEGS DREAM Challenge, 2nd place | Oct 2024
% Won 2nd place in a crowd-sourced ML challenge (team of 2) to identify high-cholesterol risk factors using multimodal data, including DNA sequence and methylation profiles.
% INTERNATIONAL CONFERENCES
% RECOMB/ISCB Conference on Regulatory & Systems - 2024, Madison, Wisconsin (invited talk) | Emerging Theoretical Approaches to Complement Single-Particle Cryo-Electron Microscopy – 2024, Trieste, Italy (poster) | DPG Spring Meeting, working group on biological physics – 2025, Regensburg, Germany (oral presentation)
% OTHER RELEVANT EXPERIENCE
% Side projects
% Self-host a personal website and various services—including a custom-built home automation system—on a Raspberry Pi
% Experienced LaTeX user, I developed a widely used book template with over 900 stars on Github
% Teaching
% Instructed 4 courses on R programming, Python programming, and developer environment setup
% Regular contributor and speaker in the local R and Python programming club events
% Event organization
% Co-organized the bi-weekly meetings of the local EMBL R programming club since Jan 2024
% Organized team-building retreat in 2022, including scientific programme and social activities
% PROFESSIONAL DEVELOPMENT
% Communication skills (2025) | Managing difficult conversations (2024) | MBTI personality types (2022)
% LANGUAGES
% Italian (mother tongue) | English (C2, professional working proficiency) | German (A1) | Chinese (A1)

% Your task is to create an updated LaTeX cv that incorporates the new content. Try to understand my personality and keep the most original bits. Use the same template as the old cv. Add new blocks if necessary, but also try to keep the existing blocks. Use the same latex template!
