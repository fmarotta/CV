\documentclass[
	DIV=15,
]{scrartcl}

\usepackage{microtype}

\usepackage{fontspec}
\setmainfont{Fira Sans}
\setsansfont{Fira Sans}

\usepackage[english]{csquotes}

\usepackage[hidelinks]{hyperref}

\usepackage{fontawesome5}

\usepackage{enumitem}
\setlist{nosep}
% \setlist[description]{labelindent=1.5em}
% \addtokomafont{descriptionlabel}{\normalfont}
\renewcommand{\labelitemi}{\raisebox{1.4pt}{\tiny ▶}}

\usepackage{biblatex}
\addbibresource{publications.bib}

\usepackage{xcolor}
\definecolor{gblue}{RGB}{18, 105, 199}
\addtokomafont{title}{\color{gblue}}
\addtokomafont{section}{\color{gblue}}
\addtokomafont{subparagraph}{\normalfont\itshape}

\RedeclareSectionCommand[beforeskip=1.5ex plus -.2ex, afterskip=.2ex plus -.2ex]{section}
\RedeclareSectionCommand[beforeskip=0.2ex plus -.2ex, afterskip=.2ex plus -.2ex]{paragraph}
\RedeclareSectionCommand[beforeskip=0.2ex plus -.2ex, afterskip=.2ex plus -.2ex, runin=false, indent=0pt]{subparagraph}

\raggedbottom
\pagestyle{empty}

\begin{document}

\begin{center}
	{\usekomafont{title} \huge Federico Marotta, PhD Candidate}\\[10px]
	\href{https://en.wikipedia.org/wiki/Turin}{\faIcon{pizza-slice} Born 1996 in Torino (Italy)}\\[4px]
	\href{mailto:fmarotta@proton.me}{\faIcon{envelope} fmarotta@proton.me}
	| \href{tel:+491625834769}{\faIcon{phone} +49 162 583 4769}\\[4px]
	\href{https://github.com/fmarotta}{\faIcon{github} GitHub}
	| \href{https://orcid.org/0000-0002-0174-3901}{\faIcon{orcid} ORCiD}
	| \href{https://www.linkedin.com/in/federico-marotta-4a6b23137/}{\faIcon{linkedin} LinkedIn}
	| \href{https://www.fmarotta.net}{\faIcon{code} Website}\\[4px]
\end{center}

\section*{Profile}

\begin{itemize}
	\item Bioinformatician with formal education in mathematics and 7 years of international academic research experience at the intersection of biology, mathematics, and computer science
	\item Strong foundation in bioinformatics, human genetics, and mathematical modelling
	\item Formal experience with DNA/RNA sequencing, design of data analysis workflows, and scientific software development in R, Python, and Julia
	\item Passionate about developing automations and leveraging technology to streamline complex processes both in and out of the workplace
	\item Personal quote: \enquote{What I cannot code, I do not understand}
\end{itemize}

\section*{Technical Skills}

\paragraph{Software development}
\begin{itemize}
	\item R---data.table, ggplot, shiny, machine learning (8-year experience, 3 own packages on CRAN)
	\item Python---data structures and algorithms, multi-threading (6-year experience)
	\item Julia---numerical computing, optimization (3-year experience)
	\item Pipeline development---Nextflow, Snakemake
	\item Databases---mongoDB, PostgreSQL
	\item NGS data analysis---variant calling, variant imputation
	\item Best practices---version control, Gitlab CI/CD, documentation and reports
\end{itemize}

\paragraph{Mathematics and Statistics}
\begin{itemize}
	\item Hypothesis testing and association studies
	\item Machine learning (supervised and unsupervised)
	\item Mathematical modelling and stochastic processes
\end{itemize}

\paragraph{Linux and High-performance Computing}
\begin{itemize}
	\item Proficient in Linux and HPC environments---slurm
	\item Bioinformatic tools---hmmer, BLAST, Foldseek
\end{itemize}

\paragraph{Domain knowledge}
\begin{itemize}
	\item Biology, statistics, bioinformatics, modelling, data analysis
\end{itemize}

\section*{Work Experience}

\paragraph{PhD candidate | Bioinformatics and mathematical modelling | 2021 -- Present}
\subparagraph{\href{https://embl.org/groups/bork}{European Molecular Biology Laboratory, Molecular and Systems Biology Unit, Germany}}
\begin{itemize}
	\item Performed bioinformatic analyses to identify the function of previously uncharacterized proteins in a clinically relevant bacterium
	\item Designed a mathematical model of a biological process, developed an optimization algorithm, and implemented it in high-quality software
	\item Built workflows with Nextflow and Snakemake for large-scale processing of DNA sequence and protein structure data
\end{itemize}

\paragraph{Graduate researcher | Human Genetics | 2017 -- 2021}
\subparagraph{\href{https://unito.it}{University of Turin, Italy}}
\begin{itemize}
	\item Processed DNA and RNA sequencing data, studying the effect of genetic variants
	\item Developed a predictive machine learning model based on DNA sequence
	\item Devised novel statistical methods to find associations between genes and disease
\end{itemize}

\paragraph{Teaching Assistant | Introduction to Physics and Statistics | 2020}
\subparagraph{\href{https://unito.it}{University of Turin, Italy}}
\begin{itemize}
	\item Developed training material, including homework and explanatory visualizations
	\item Provided support to students by addressing questions and clarifying concepts
\end{itemize}

\section*{Education}

\paragraph{PhD candidate in Bioinformatics}
\subparagraph{\href{https://embl.org}{Joint degree European Molecular Biology Laboratory and Heidelberg University, Germany}}
\begin{itemize}
	\item Dissertation: Characterization, evolution, and dynamics of cryoET-derived macromolecular assemblies in \textit{Mycoplasma pneumoniae}
	% \item[Supervisors and mentors:] Prof. Peer Bork, Dr. Julia Mahamid, Dr. Maria Zimmermann-Kogadeeva, Prof. Sophia Rudorf
	\item Supervisor and mentor: Prof. Peer Bork, Prof. Sophia Rudorf
\end{itemize}

\paragraph{M.Sc. in Stochastics and Data Science (Applied Mathematics) | Grade 110/110 cum laude}
\subparagraph{\href{https://unito.it}{University of Turin, Italy}}
\begin{itemize}
	\item Dissertation: The effect of genetic variants on complex diseases
	% \item Grade: 110/110 cum laude
	\item Supervisors: Prof. Paolo Provero, Prof. Silvia Montagna
\end{itemize}

\paragraph{B.Sc. in Biotechnology | Grade 110/110 cum laude}
\subparagraph{\href{https://unito.it}{University of Turin, Italy}}
\begin{itemize}
	\item Dissertation: Transcriptome-wide association studies
	% \item Grade: 110/110 cum laude
	\item Supervisors: Prof. Paolo Provero
\end{itemize}

\section*{Awards}

\paragraph{PEGS DREAM Challenge, 2nd place | 2024}
\begin{itemize}
	\item Won 2nd place in a crowd-sourced ML challenge (team of 2) to identify high-cholesterol risk factors using multimodal data
	\item Gave a talk at the RECOMB/ISCB Conference on Regulatory \& Systems Genomics in Madison, Wisconsin (\href{https://www.youtube.com/watch?v=QCDiowWxeJM}{YouTube video})
\end{itemize}

\section*{Selected Publications and International Conferences}

\begin{itemize}
	% \item \textbf{RECOMB/ISCB Conference on Regulatory \& Systems Genomics} --- 2024, Madison, Wisconsin (invited talk)
	\item \citetitle{zhao2024thanos} (journal article + R package)
	\item \citetitle{marotta2020fplyr} (preprint + R package)
	\item \citetitle{jensen2025cell} (preprint)
	\item \citetitle{marotta2021prediction} (preprint + code)
	% \item \enquote{Emerging Theoretical Approaches to Complement Single-Particle Cryo-Electron Microscopy} – 2024, Trieste, Italy (conference poster)
	\item \enquote{DPG Spring Meeting}, working group on biological physics --- 2025, Regensburg, Germany (oral presentation)
\end{itemize}

\section*{Relevant Experience}

\paragraph{Side projects}
\begin{itemize}
	\item Self-host a personal website and various services---including a custom-built home automation system---on a Raspberry Pi
	\item Experienced LaTeX user, I designed a widely used book template with over 900 stars on Github
\end{itemize}

\paragraph{Teaching and Event organization}
\begin{itemize}
	\item Instructed 4 courses on R programming, Python programming, and developer environment setup
	\item Regular contributor and speaker in the local R and Python programming club events
	\item Co-organized the bi-weekly meetings of the local EMBL R programming club since Jan 2024
\end{itemize}

 % Organized team-building retreat in 2022, including scientific program and social activities.

\section*{Personal and Professional Development}

\begin{itemize}
	\item Attended courses on \enquote{Communication skills} (2025), \enquote{Managing difficult conversations} (2024), \enquote{MBTI personality types} (2022), and \enquote{Assert yourself} (2022)
	\item Besides programming languages, I dabble in some human languages including Italian (native), English (professional), German (A1), and Chinese (A1)
\end{itemize}

\end{document}
